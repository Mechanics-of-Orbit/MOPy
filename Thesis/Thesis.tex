\documentclass[12pt]{article}
\usepackage{newtxtext}
\usepackage[T1]{fontenc}
\usepackage{etoolbox}
\usepackage{booktabs}



\makeatletter
%\patchcmd{<cmd>}{<search>}{<replace>}{<success>}{<failure>}
\patchcmd{\@makechapterhead}{\huge}{\large}{}{}% for \chapter
\patchcmd{\@makechapterhead}{\Huge}{\large}{}{}% for \chapter
\patchcmd{\@makeschapterhead}{\Huge}{\large}{}{}% for \chapter*
\patchcmd{\@makeschapterhead}{\HUGE}{\large}{}{}
\makeatother

\usepackage[top=1in, bottom=1in, left= 1in, right=1in]{geometry}
\usepackage{amsmath}
\usepackage{enumitem}
\usepackage[english]{babel}
\usepackage{hyperref}
\usepackage[utf8]{inputenc}
% \usepackage{xcolor}
\usepackage[autostyle]{csquotes}
\usepackage{graphicx}
\usepackage[stable]{footmisc}
\usepackage{subfig}
\usepackage{listings}
\usepackage{floatrow}
\usepackage{textcomp}
\usepackage{multirow}
\usepackage{tabulary}
\newcolumntype{K}[1]{>{\centering\arraybackslash}p{#1}}
\usepackage[table]{xcolor}
\newfloatcommand{capbtabbox}{table}[][\FBwidth]
\usepackage{framed}

\definecolor{codegreen}{rgb}{0,0.6,0}
\definecolor{codegray}{rgb}{0.5,0.5,0.5}
\definecolor{codepurple}{rgb}{0.58,0,0.82}
\definecolor{backcolour}{rgb}{0.95,0.95,0.92}

\lstdefinestyle{mystyle}{
    backgroundcolor=\color{backcolour},   
    commentstyle=\color{codegreen},
    keywordstyle=\color{magenta},
    numberstyle=\tiny\color{codegray},
    stringstyle=\color{codepurple},
    basicstyle=\ttfamily\footnotesize,
    breakatwhitespace=false,         
    breaklines=true,                 
    captionpos=b,                    
    keepspaces=true,                 
    numbers=left,                    
    numbersep=5pt,                  
    showspaces=false,                
    showstringspaces=false,
    showtabs=false,                  
    tabsize=2
}
\lstset{style=mystyle}
\setlength{\parindent}{4em}
\setlength{\parskip}{1em}
\setcounter{tocdepth}{3} % Show sections

\usepackage{pgf}
\usepackage{pgfpages}

\pgfpagesdeclarelayout{boxed}
{
  \edef\pgfpageoptionborder{2pt}
}
{
  \pgfpagesphysicalpageoptions
  {%
    logical pages=1,%
  }
  \pgfpageslogicalpageoptions{1}
  {
    border code=\pgfsetlinewidth{2pt}\pgfstroke,%
    border shrink=\pgfpageoptionborder,%
    resized width=.95\pgfphysicalwidth,%
    resized height=.95\pgfphysicalheight,%
    center=\pgfpoint{.5\pgfphysicalwidth}{.5\pgfphysicalheight}%
  }%
}

\pgfpagesuselayout{boxed}


%\title{}
%\author{ }
%\date{ }
\begin{document}


\begin{center}
\normalsize{A dissertation report on} \\ 
\vspace*{1em}
\Large{\textbf{\enquote{Interactive Learning Platform for Orbital Mechanics Using Python}}}\\ \vspace{0.5em}
\normalsize{Submitted to}\\ \vspace{1em}
\includegraphics[scale=0.7]{AU.png}\\ \vspace{1em}
\normalsize{In partial fulfilment of the requirements for the award of degree} \\
\Large{\textbf{Bachelor of Technology}}\\
\Large{\textbf{In}}\\
\Large{\textbf{Aerospace Engineering}}\\ \vspace*{1em}
\normalsize Submitted by: \vspace*{1em} \\
\normalsize
\begin{tabular}{cc}
\textbf{Ramkiran L.} & \textbf{Manjunath} \\ 
17030141AE007 & 17030141AE009 \\ 
{{{\fontfamily{ptm}\selectfont
lramkiranBTECH17@ced.alliance.edu.in}}} & {{{\fontfamily{ptm}\selectfont
manjunathBTECH17@ced.alliance.edu.in}}} \vspace*{1em} \\ 
\textbf{Monisha Patel A.} & \textbf{Thoshitha R. Kumar} \\ 
17030141AE013 & 17030141AE027 \\ 
{{{\fontfamily{ptm}\selectfont
pamonishaBTECH17@ced.alliance.edu.in}}} & {{{\fontfamily{ptm}\selectfont
kuthoshithaBTECH17@ced.alliance.edu.in}}} \vspace*{1em} \\ 
\end{tabular} 
\normalsize
Under the guidance of\\
\begin{tabular}{cc}
\textbf{Prof. Gisa G.S.} & \textbf{Prof. Yadu Krishnan} \\
Assistant Professor & Assistant Professor \\
Department of Aerospace Engineering,& Department of Aerospace Engineering,\\
Alliance College of Engineering and Design, & Alliance College of Engineering and Design,\\
Alliance University, Bengaluru. & Alliance University, Bengaluru. \vspace*{1em}\\
\end{tabular}
\textbf{Department of Aerospace Engineering}\\
\textbf{Alliance College of Engineering and Design}\\
\textbf{Alliance University, Bengaluru - 562106} \\
\textbf{Batch - 2017-21}\\
\textbf{Year - 2021}
\end{center}
\thispagestyle{empty}
\newpage
\begin{center}
\includegraphics[scale=0.7]{AUText.png} \vspace*{2em}\\
\textbf{CERTIFICATE}
\end{center}
This is to certify that \textbf{Mr. Ramkiran L. (17030141AE007), Mr. Manjunath (17030141AE009), Ms. Monisha Patel A. (17030141AE012)} and \textbf{Ms. Thoshitha R. Kumar (17030141AE027)} students of \textbf{Aerospace Engineering, Bachelor of Technology 2017-21} batch at \textbf{Alliance College of Engineering and Design (ACED), Alliance University, Bengaluru} has completed the project report titled \textit{\textbf{\enquote{Interactive Learning Platform for Orbital Mechanics Using Python}}} under our guidance in partial fulfillment for the award of Bachelor of Technology degree in Aerospace Engineering, Alliance University, Bangalore during the year 2020-2021. \vspace{1em}\\
\begin{center}
\begin{tabular}{K{7cm} K{0.5cm} K{7cm}}
\underline{\hspace{2.9cm}} & & \underline{\hspace{4.3cm}} \\ 
\textbf{Prof. Gisa G.S} & & \textbf{Prof. Yadu Krishnan} \vspace{1em}\\ 
Internal Guide & &  Internal Guide\\ 
Department of Aerospace Engineering & & Department of Aerospace Engineering \\
ACED, Alliance University & & ACED, Alliance University \\ 
Bengaluru & & Bengaluru \vspace{3em}\\ 
\underline{\hspace{5cm}}&&\underline{\hspace{3.5cm}}\\ 
\textbf{Prof. Velmurugarajan K.}&&\textbf{Dr. Reeba Korah}\vspace{1em} \\ 
Head of the Department&&Interim Dean \\ 
Department of Aerospace Engineering&&Department of Aerospace Engineering\\
ACED, Alliance University&&ACED, Alliance University\\ 
Bengaluru&&Bengaluru\vspace{2em}\\ 
\end{tabular} 
\end{center}
\textbf{External Viva}
\begin{center}
\begin{tabular}{c K{7cm} K{7cm}}
 & \textbf{Name of Examiners} & \textbf{Signature with date} \\
\textbf{1.} & & \\ 
&&\\
\textbf{2.} & & \\
\end{tabular}
\end{center}
\thispagestyle{empty}
\newpage
\begin{center}
\Large \textbf{DECLARATION}
\end{center}
\normalsize
\hspace{4em}We, Ramkiran L, Manjunath, Monisha Patel A, Thoshitha R. Kumar students of 8\textsuperscript{th} Semester Bachelor of Technology in Aerospace Engineering, Alliance College of Engineering and Design (ACED), Alliance University, Bengaluru, hereby declare that the entire project work entitled \enquote{\textbf{Interactive Learning Platform for Orbital Mechanics Using Python}} is an authentic record of the work that has been carried out independently by us during final year of our B.Tech at ACED, under the esteemed guidance of \textbf{Prof. Gisa G.S} and \textbf{Prof. Yadu Krishnan S.}, Assistant Professors, Department of Aerospace Engineering, Alliance college of Engineering and Design, Alliance University. \par

This project report is submitted in partial fulfillment of requirements for the award of the degree of Bachelor of Technology in Aerospace Engineering. The results embodied in this dissertation are original and it has not been submitted in part or full for any degree in any University. \vspace{4em}\\
\textbf{Place:} Bengaluru\\
\textbf{Date:}17/062021 \vspace{5em}\\

\begin{center}
\begin{tabular}{K{7.5cm} K{7.5cm}}
\underline{\hspace{2.5cm}} & \underline{\hspace{2.5cm}} \\ 
\textbf{Ramkiran.L} & \textbf{Manjunath} \\ 
\textbf{17030141AE007} & \textbf{17030141AE009} \vspace{3em}\\ 
\underline{\hspace{3cm}} & \underline{\hspace{4cm}} \\ 
\textbf{Monisha Patel A.} & \textbf{Thoshitha R. Kumar} \\ 
\textbf{17030141AE012} & \textbf{17030141AE027}
\end{tabular} 
\end{center}
\thispagestyle{empty}
\newpage
\begin{center}
\Large \textbf{ACKNOWLEDGEMENT}
\end{center}
\normalsize
\hspace{4em}The satisfaction that accompanies the successful completion of any task would be incomplete without the mention of the people, who are responsible for the completion of the project and who made it possible.\par

We take this opportunity to thank our beloved Interim Dean \textbf{Dr. Reeba Korah}, ACED, Alliance University, Bangalore for providing excellent academic environment in the college and her never--ending support to the B-Tech program.\par

We would like to convey our sincere gratitude to \textbf{Prof. K. Velmurugarajan}, Head of Department of Aerospace Engineering, ACED, Alliance University, Bangalore. \par

We would like to thank our internal guide \textbf{Prof. Gisa G.S} and \textbf{Prof. Yadu Krishnan S.}, Assistant Professors, Department of Aerospace Engineering, ACED, Alliance University, Bangalore for her support and encouragement given to carry out the project. \par

We would also like to thank our college staff members and well-wishers who directly or indirectly helped, motivated to complete this project successfully. \par

Lastly, we thank God almighty, our family, professors and friends for their constant encouragement without which this project would not have been possible.\\
\thispagestyle{empty}
\newpage
\begin{center}
\Large Abstract
\end{center}
\normalsize
\hspace{4em}Orbital mechanics is the study of the motions of artificial satellites and space vehicles moving under the influence of forces. It plays a vital role in planning space missions in various aspects such as designing orbital trajectory for various missions, calculating lagrangian points etc., and can find many engineering applications  including ascent trajectories, reentry and landing, rendezvous computations, lunar and interplanetary trajectories. 

In recent years, as there is a lot of progress unfolding in space industry, many aspiring students are keen on gaining knowledge and pursue careers in space industry. As such, strong foundations of the fundamentals are required for the students for them to get ahead in the field. There are lots of tools like STK, FreeFlyer etc., for learning but these don't start from the bare minimum of the concepts. This led to the idea of developing of MOPy.

Mechanics of Orbit using Python(MOPy) is learning tool designed and developed with the purpose of introducing the core concepts of Orbital Mechanics. MOPy uses a interactive UI with tool tips and a 3D Environment with an interactive virtual universe. The 3D Environment helps the learner to visualize the concepts in a much accessible and easier way.

\thispagestyle{empty}
\newpage
\thispagestyle{empty}
\tableofcontents
\listoffigures
\thispagestyle{empty}
\listoftables
\thispagestyle{empty}
\newpage
\setcounter{page}{1}
\section{Introduction}
\hspace{4em}MOPy, a learning tool designed to learn and practice various orbital mechanics concepts. It is designed in a way such that it can be a user-friendly tool that can be operated with ease even by the user who has very limited knowledge about the concepts of orbital mechanics. It provides users to learn about a particular concept with a brief explanation so the user can gain the theoretical knowledge required through visualizations. The sophisticated 3D environment benefits the user to visualize the fundamental concepts easily. It assists the user to verify manually calculated data. It serves as advanced virtual calculator. MOPy is an open source software with GNU-GPL v3 license\cite{lic} which anyone can use or work with. It runs on windows platforms at present.
\begin{figure}[H]
\centering
\includegraphics[scale=0.18]{mopy.png}
\caption{MOPy} \label{mopy}
\end{figure}
\subsection{List of Features}
\begin{center}
\begin{table}[H]
{\rowcolors{2}{white}{gray!50}
\begin{tabular}{c|K{7.5cm}}
\hline 
Sl. No & \textbf{Section}\\ 
\hline 
1 & Calculation of Orbital Elements\\ 
\hline 
2 & 2D and 3D orbit \\ 
\hline 
3 & Various Parameters at any given point \\ 
\hline 
4 & 2D and 3D Orbits\\ 
\hline 
5 & Calculation of Julian Day \\ 
\hline 
5 & Euler Angles\\ 
\hline 
6 & Sphere Of Influence \\ 
\hline 
7 & Sensitivity Analysis \\ 
\hline 
8 & Position of one Spacecraft w.r.t Another\\ 
\hline 
9 & Calculation of State and Velocity Vector\\ 
\hline
10 & Orbital Transfer \\
\hline
\end{tabular}}
\caption{\label{tab: features}List of Features present in MOPy}
\end{table}
\end{center}
\subsection{Libraries Used}
\begin{enumerate}
\item \textbf{NumPy}:  This brings MATLAB like functionality of using Matrices and their operations to python. This enables us to do a lot of stuff without much hassle.
\item \textbf{SciPy} - This enables us to add many features involving more complex computing scenarios as it has features for scientific and technical computing. For example, it has different kinds of solvers for integration which we can use for solving acceleration vector equation to obtain the position vector for an orbit.
\item \textbf{Matplotlib} - This is a plotting tool that is a extension on NumPy that gives the functionality of plotting many different kinds of graph. This library is somewhat similar to the plotting feature of MATLAB.
\item Panda3D - This is a Game Engine based on C++ that takes in syntax from both C++ and Python. This provides real-time 3D visualizations and simulations based on the code.
\item \textbf{SQLite3} - The entire details of the planetary bodies like the orbital elements, planetary ephemeris and others are stored in a local database. SQLite is used as it enables the offline functionality.
\item \textbf{Qt Deisgner} - This enables MATLAB's App Designer like feature of dragging and dropping the UI elements and creating the GUI. This is based on Qt, a cross platform GUI toolkit developed by the Qt Company
\item \textbf{PyQt5 $\&$ PySide2} - These both are the python binding libraries of Qt.
\item \textbf{PyInstaller} - This library lets us convert our python code(.py) into executable file(.exe)
\end{enumerate}
\subsection{Market Research}
There are mainly two kinds of softwares.
\begin{enumerate}
\item Simulation Based programs like STK, FreeFlyer etc.,
\item Sandbox Based programs like Universe SandBox, Kerbal Space Program etc.,
\end{enumerate}
\hspace{4em}The simulation based programs are mainly used to simulate missions and solve problems based on the instance. Both the applications given in the example are used in the industry for all kinds of missions, ranging from very small scale missions that are performed by the students to complicated missions that are performed by NASA and ISRO.

The Sandbox based programs are the stuff that are run by the physics engine that are baked into it. They use a 3D visualization toolkit or engine which lets the user easily interact with the UI, and change the parameters directly from the environment.

In the analysis done by Morgan Stanley named \enquote{Investing in Space Exploration}, it is stated as - The revenue generated by the global space industry may increase to more than $\$$1 trillion by 2040. \cite{morgan}
\begin{figure}[H]
\centering
\includegraphics[scale=0.25]{morganstanley.png}
\caption{Global Market Trend according to Morgan Stanley.} \label{morgangraph}
\end{figure}
A report by Antrix and PwC, it is stated that the indian space sector can become a $\$$50 Billion industry, or about one per cent of India's projected $\$5$ Trillion economy, by 2024 from the current $\$7$ billion, according to a report by the Antrix and PwC.\cite{indiaspace}
\begin{figure}[H]
\centering
\includegraphics[scale=1]{pwc.png}
\caption{Indian Space Budget over the years.} \label{pwc}
\end{figure}
\subsection{Objective}
The development in the field of space technology is constantly increasing as shown in the aforementioned studies.
Such being the case, many students are showing interest to learn more and more about space technology and its related concepts. Considering all such possibilities, we have come up with an idea to develop a learning tool beneficial to learn more about Orbital Mechanics 
The main objectives of this project are as follows:
\begin{enumerate}
\item Design and develop a software to learn concepts and  solve Problems related orbital mechanics to understand the basics.
\item Provide a user friendly learning tool such that the user can operate even with the minimum knowledge about the concepts of orbital mechanics.
\item Help user to visualize the virtual view of the space mission.
\end{enumerate}
\subsection{Front End Development}
\subsubsection{Introduction}
Front-end development deals with the Graphical User-Interface aspect of the software. It is the key developmental process that defines how the user experiences the features we have developed. The interface between the user and the back end code is GUI. The inputs from the user is taken from the GUI. So the design must be intuitive and clear. There are many libraries that can be used to develop a GUI like PyQt, Pyside, Kivy, Tkinter, etc. In our case, we have opted for PyQt5, Pyside2 and designed GUI in Qt-Designer. Then linked the Back-End scripts through the Integrated Development Environment(IDE) by Microsoft i.e, Visual Studio Code.
\subsubsection{Home Page}
When the application opens the Home Page will load. In it, there is a Dropdown box containing all the features available, from which the user can choose which feature they want to use. When the user selects any of the features from the drop-down and clicks on the go button at the bottom, they are navigated to that screen where they can use the feature they selected. And then if they want to navigate back to the Home-Page they can click on the Home button provided at the top left corner of the screen. All the features are listed in the upper-mentioned table.
\begin{figure}[H]
\centering
\includegraphics[scale=0.6]{homepage.png}
\caption{Home Page of MOPy} \label{home}
\end{figure}
\section{Detailed Explanation of Each Feature}
\subsection{Calculation of Orbital Elements}
This can convert state velocity vector into orbital elements and vice versa. The inputs are as follows
\begin{table}[H]
\centering
\begin{tabular}{@{}cl@{}}
\toprule
\multirow{2}{*}{Inputs} & State and Velocity Vectors            \\ \cmidrule(l){2-2} 
                                 & \multicolumn{1}{l}{Orbital Elements} \\ \midrule
\multicolumn{1}{r}{\multirow{2}{*}{Outputs}} & \multicolumn{1}{l}{Orbital Elements} \\ \cmidrule(l){2-2} 
\multicolumn{1}{r}{}           & State and Velocity Vectors            \\ \bottomrule
\end{tabular}\caption{Inputs and outputs for Calculation of Orbital Elements}
\end{table}
\begin{figure}[H]
\begin{floatrow}
\ffigbox{%
  \includegraphics[scale=0.3]{COE.png}
}{
  \caption{Classical Orbital Elements} \label{COE}
}
\ffigbox{%
  \includegraphics[scale=0.5]{AOE.png}
}{
  \caption{Alternate Orbital Elements} \label{AOE}
}
\end{floatrow}
\end{figure}


\subsubsection{Algorithm}
\begin{lstlisting}[language=python, caption=Calculation of Orbital Elements]
# Orbital Constants
h_vec = cross(pos_vec, vel_vec)
n_vec = cross(cls.K,h_vec)
# Classical Orbital Elements
sma = 1/((2/norm(pos_vec))-((norm(vel_vec)*norm(vel_vec))/mu))
e_vec = (multi((norm(vel_vec)*norm(vel_vec)-(mu/norm(pos_vec))),pos_vec)- multi(dot(pos_vec,vel_vec),vel_vec))/(mu)
inc = (acos((dot(h_vec, cls.K))/norm(h_vec))) * 180/pi
ohm = (acos((dot(cls.I,n_vec))/norm(n_vec)))
nu = (acos((dot(e_vec,pos_vec))/(norm(e_vec)*norm(pos_vec))))
omega = (acos((dot(n_vec,e_vec))/multi(norm(n_vec),norm(e_vec))))
# Alternate Orbital Elements
Long_of_peri_pi = acos(dot(cls.I,e_vec)/(norm(cls.I)*norm(e_vec)))
Tr_long_l = acos(dot(cls.I,pos_vec)/(norm(pos_vec)*norm(cls.I)))
Arg_of_lattitude_u = acos(dot(n_vec,pos_vec)/(norm(n_vec)*norm(pos_vec)))
\end{lstlisting}

\subsection{2D and 3D Orbits}

Based on the inputs given the application can plot either 2D or 3D or both. If the inputs are just Semi major axis and eccentricity or Radius of perigee and apogee or, $r_1,\;v_1,\;\gamma_1$ then the resultant plot will be in the perifocal frame. If the inputs are state and velocity vector or orbital elements then the resultant will be a 3D orbit. 
\begin{table}[H]
\centering
\begin{tabular}{@{}cl@{}}
\toprule
\multirow{2}{*}{\textbf{Inputs}}                      & For 2D - $a,e/r_a,r_p/ r_1, v_1, \gamma_1$ \\ \cmidrule(l){2-2} 
                                             & For 3D - Orbital Elements, State Vectors   \\ \midrule
\multicolumn{1}{r}{\multirow{2}{*}{\textbf{Outputs}}} & Orbit in the Perifocal Frame               \\ \cmidrule(l){2-2} 
\multicolumn{1}{r}{}                         & Orbit in a virtual 3D Environment          \\ \bottomrule
\end{tabular}
\caption{I/O for 2D and 3D orbits}
\label{o23}
\end{table}
\subsection{Euler Angle}

The inputs for this are as in table(\ref{eadcm}). This code can be fed the model in 3D environment and the orientation of the model can be manipulated with this.
\begin{table}[H]
\centering
\begin{tabular}{@{}cl@{}}
\toprule
\multirow{2}{*}{\textbf{Inputs}}                      & Directional Cosine Matrix \\ \cmidrule(l){2-2} 
                                             & Euler Angles              \\ \midrule
\multicolumn{1}{r}{\multirow{2}{*}{\textbf{Outputs}}} & Euler Angles              \\ \cmidrule(l){2-2} 
\multicolumn{1}{r}{}                         & Directional Cosine Matrix \\ \bottomrule
\end{tabular}
\caption{I/O for conversion between Euler angle and DCM}
\label{eadcm}
\end{table}
\subsection{Sphere of Influence}
In this section, the user can either input the values or interact with the model in the 3D environment to see the corresponding output.
\begin{table}[H]
\centering
\begin{tabular}{@{}rl@{}}
\toprule
\multicolumn{1}{c}{\textbf{Inputs}} & Minor Body                     \\ \midrule
\multirow{2}{*}{\textbf{Outputs}}   & Radius of SOI in desired units \\ \cmidrule(l){2-2} 
                           & 3D visualization of sphere of influence in virtual environment                         \\ \bottomrule
\end{tabular}
\caption{I/O for Sphere of Influence}
\label{soi}
\end{table}
\subsection{Orbital Transfer}
This feature simulates the orbital transfer using various numerical methods. For now, this can perform Hohmann Transfer. The Inputs and outputs are as in table(\ref{tab:ot})
\begin{table}[H]
\centering
\begin{tabular}{@{}cl@{}}
\toprule
\multicolumn{1}{c}{\textbf{Inputs}} & Minor Body, Major Body, Orbital parameters of both initial and final orbit.                     \\ \midrule
\multirow{2}{*}{\textbf{Outputs}}   & Values such as Radius of apogee, Radius of perigee, DeltaV, Time-period of the orbit etc \\ \cmidrule(l){2-2} 
                           & 3D visualization of desired orbital transfer \\ \bottomrule
\end{tabular}
\caption{I/O for Orbital transfer}
\label{tab:ot}
\end{table}
\subsection{Calculation of Julian Day}
In this section the user can choose between Gregorian calendar and Julian calendar and with the other inputs they can obtain the Julian day of the corresponding date. 
\begin{table}[H]
\centering
\begin{tabular}{@{}cl@{}}
\toprule
\textbf{Inputs}  & YYYY-MM-DD , hh:mm:ss, Type of calender \\ \midrule
\textbf{Outputs} & Julian-day  \\ \bottomrule                                              
\end{tabular}
\caption{I/O for Julian-day}
\label{tab:jd}
\end{table}
\subsection{Calculation of Parameters of the Orbit}
The user can obtain various parameters by giving the details that they know of. If necessary they can plot the orbit too.
\begin{table}[H]
\centering
\begin{tabular}{@{}rl@{}}
\toprule
\multicolumn{1}{c}{\textbf{Inputs}} & $a,e/r_a, r_p/r_1,\;v_1,\;\gamma_1$/ Orbital Elements/State and velocity vectors \\ \midrule
\multirow{2}{*}{\textbf{Outputs}}   & $\mu,\;,h\;,\epsilon$, Forces and Velocity at significant position               \\ \cmidrule(l){2-2} 
                           & Mean Motion, Time Period                                                         \\ \bottomrule
\end{tabular}
\caption{I/O for Various Parameters of the Orbit}
\label{vpco}
\end{table}
\subsection{Sensitivity Analysis}
User has to select the type of calendar and the in the input section they have to select date and time and accuracy of digits and then by clicking on calculate button we will get Julian days in the output section. This takes in the required values an outputs how much of an error will that small change in the velocity or radius in the 3D environment.

\begin{table}[H]
\begin{tabular}{@{}cl@{}}
\toprule
\textbf{Input}  & State Vector, Velocity Vector and two delta-v with slight difference                                                                      \\ \midrule
\textbf{Output} & \begin{tabular}[c]{@{}l@{}}Percentage difference caused to the final orbit parameters due to slight \\ difference in delta-v\end{tabular} \\ \bottomrule
\end{tabular}\caption{I/O for Sensitivity Analysis}
\end{table}

\subsection{Position of One Spacecraft Relative To Another}
Based on the inputs of the user the relative velocity and orbit can be visualized in this section.
\begin{table}[H]
\begin{tabular}{@{}cl@{}}
\toprule
\textbf{Input}  & Major Body, State Vector and Velocity Vector of Minor Body    \\ \midrule
\textbf{Output} & Graph Showing the Minor bodies in the orbit around Major Body \\ \bottomrule
\end{tabular}\caption{I/O for Position of One Spacecraft Relative To Another}
\end{table}
\subsection{Lagrangian Points}
The Lagrangian points are points near two large orbiting bodies, the two objects exert an unbalanced gravitational force at a point, altering the orbit of whatever is at that point. At the Lagrange points, the gravitational forces of the two large bodies and the centrifugal force balance each other, The inputs and outputs are as follows.
\begin{table}[H]
\begin{tabular}{@{}cc@{}}
\toprule
\textbf{Input}  & Major Body, Minor Body and Distance Between them           \\ \midrule
\textbf{Output} & Lagrangian points polar coordinates and Graph showing them \\ \bottomrule
\end{tabular}\caption{I/O for Lagrangian Points}
\end{table}

\section{Database Management System}
\subsection{Introduction}
Database is a collection of set of related data or information of any particular concept, generally stored using any electronic gadget . Database Management system is a system which uses integrated software to connect frontend user to connect to the database to access , modify and manage the data stored in it.  
There are two types of Databases:
\begin{enumerate}
\item Relational Database - Organizes data into one or more table. each table comprising of number of rows and columns  with each row identifiable with unique key.
\item Non - Relational database - Organizes data in any form other than table. Such as graphs, flexible tables, documents etc.
\end{enumerate}
In this project we have used Relational database management system with structured Query Language (SQL) to interact with the RDBMS. SQLite is the database engine used to run SQL query to perform CRUD ( Create , Retrieve, Update or Delete) operation in the database.

\subsection{Scripts for performing CRUD Operation}
\begin{lstlisting}[language=python, caption=Script file for performing CRUD Operation]
# import sqlite in ide 
import sqlite3

# Create a database or connect to one 
conn= sqlite3.connect('appdatabase.db')

# Create a Cursor
c= conn.cursor()

# Creat a table
c.execute(""" CREATE TABLE table_name( 
        column_name datatype,
        column_name datatype
     )""")

# Insert data into table
x = [('column 1', 'column 2')
     ('column 1', 'column 2')
    ]
c.executemany( "INSERT INTO table_name VALUES(?,?)",x)

#Retrieve or read data 
c.execute("SELECT * FROM table_name")
print(c.fetchall())

# commit data to the database
conn.commit()
\end{lstlisting}
This database contains the data of planetary bodies such as Mass, radius, density, gravity, temperature, orbital parameters, number of moons etc., of all planets of our solar system.
\section{Conclusion}
The outcome of the project is the lessons we learnt during the project. Communication is the key for collaborative work. This is one of the main aspect that we learned during this project. The other aspects that we were able to learn are:
\begin{enumerate}
\item Project Management
\item Time Management
\item Prioritize the list that is at hand based on various parameters
\end{enumerate}
\section{Future Scope}
\hspace{4em}MOPy is a open source application with GNU - GPL v3 License, with this anyone who is interested in using this application can freely download and use it, and even modify the contents to their requirements. If they wish to contribute to this application they can fork the repository on GitHub[3] and add their contribution by sending a pull request. The reason GNU-GPL v3 license is used is that we aim to keep a quality control to the features that gets added to the application. We are always open to contributors.

There is a list of features with corresponding priorities that are planned to be added down the line. This will be constantly updated indefinitely. And if the user is not able to add a feature then they can request a feature that would be added to the list with a appropriate priority. There is discord server in which anyone can join and hold discussions with other participants. Contributors can hold discussions about the feature they are planning to add, others can discuss about the concepts and request features to be added. 

As of now this runs on Windows only. This could be made to run on other platforms and even as a web application which would negate the need of a moderately powerful hardware. The database could be expanded with much more data than present.

This entire code can be converted into a library that anyone can download from PyPI. This would enable the user to utilize the features directly from the command line or their python projects. This would simplify their projects.
\begin{thebibliography}{99}
	\bibitem{TB-1}{Bate, Roger R., Donald D. Muller and Jerry E. White},{ \textit{\enquote{Fundamentals Of Astrodynamics}}},{New York, NY},{ Dover Publications},{1971}.
	\bibitem{morgan}{\textit{\enquote{Space: Investing in the Final Frontier}}},{Jul, 24$^{th}$, 2020}, { Retrieved on 25$^{th}$ May 2021 from Morgan Stanley},{\href{https://www.morganstanley.com/ideas/investing-in-space}{ MorganStanley.com}}
	\bibitem{indiaspace}{\textit{\enquote{Preparing to Scale New Heights: Privatisation of India's commercial Space Sector}}},{ Retrieved on 25$^{th}$ May 2021 from Morgan Stanley},{\href{https://www.pwc.in/research-insights/2020/preparing-to-scale-new-heights.html}{ pwc.in}}
	\bibitem{lic}{GNU General Public License version-3},{ Open Source Initiative},{\href{https://opensource.org/licenses/gpl-3.0.html}{ OpenSourceInitiative.org}}
	\bibitem{code}{\textit{\enquote{Mechanics of Orbit Using Python(MOPy)}}}, {\href{https://github.com/iamlrk/MOPy}{GitHub Repository}}.
		\bibitem{py}{Python Documentation},{\href{https://docs.python.org}{ python.org}}
	\bibitem{pyqt}{PyQt5 Documentation},{\href{https://www.riverbankcomputing.com/static/Docs/PyQt5/}{ riverbankcomputing.com}}
	\bibitem{pyside}{PySide2 Documentation},{\href{https://srinikom.github.io/pyside-docs/pysideapi2.html}{ srinikom.github.io}}
	\bibitem{qtd}{Qt Designer Documentation},{\href{https://doc.qt.io/qt-5/qtdesigner-manual.html}{ doc.qt.io}}	
	\bibitem{numpy}{NumPy Documentation}, {\href{https://numpy.org/doc}{ NumPy.org}}
	\bibitem{scipy}{SciPy Documentation}, {\href{https://scipy.org/doc}{ SciPy.org}}
	\bibitem{sqlite}{NumPy Documentation}, {\href{https://numpy.org/doc}{ NumPy.org}}
\end{thebibliography}
\end{document}
