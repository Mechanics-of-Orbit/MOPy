\documentclass[12pt]{article}
\usepackage{times}
\usepackage{etoolbox}

\makeatletter
%\patchcmd{<cmd>}{<search>}{<replace>}{<success>}{<failure>}
\patchcmd{\@makechapterhead}{\huge}{\large}{}{}% for \chapter
\patchcmd{\@makechapterhead}{\Huge}{\large}{}{}% for \chapter
\patchcmd{\@makeschapterhead}{\Huge}{\large}{}{}% for \chapter*
\patchcmd{\@makeschapterhead}{\HUGE}{\large}{}{}
\makeatother

\usepackage[top=1in, bottom=1in, left= 1in, right=1in]{geometry}
\usepackage{amsmath}
\usepackage{enumitem}
\usepackage[english]{babel}
\usepackage{hyperref}
\usepackage[utf8]{inputenc}
% \usepackage{xcolor}
\usepackage[autostyle]{csquotes}
\usepackage{graphicx}
\usepackage[stable]{footmisc}
\usepackage{subfig}
\usepackage{listings}
\usepackage{floatrow}
\usepackage{textcomp}
\usepackage{tabulary}
\newcolumntype{K}[1]{>{\centering\arraybackslash}p{#1}}
\usepackage[table]{xcolor}
\newfloatcommand{capbtabbox}{table}[][\FBwidth]
\usepackage{framed}

\definecolor{codegreen}{rgb}{0,0.6,0}
\definecolor{codegray}{rgb}{0.5,0.5,0.5}
\definecolor{codepurple}{rgb}{0.58,0,0.82}
\definecolor{backcolour}{rgb}{0.95,0.95,0.92}

\lstdefinestyle{mystyle}{
    backgroundcolor=\color{backcolour},   
    commentstyle=\color{codegreen},
    keywordstyle=\color{magenta},
    numberstyle=\tiny\color{codegray},
    stringstyle=\color{codepurple},
    basicstyle=\ttfamily\footnotesize,
    breakatwhitespace=false,         
    breaklines=true,                 
    captionpos=b,                    
    keepspaces=true,                 
    numbers=left,                    
    numbersep=5pt,                  
    showspaces=false,                
    showstringspaces=false,
    showtabs=false,                  
    tabsize=2
}
\lstset{style=mystyle}
\setlength{\parindent}{4em}
\setlength{\parskip}{1em}
\setcounter{tocdepth}{3} % Show sections




%\title{}
%\author{ }
%\date{ }
\begin{document}
\begin{center}
\normalsize{A dissertation report on} \\ 
\vspace*{1em}
\Large{\textbf{\enquote{Mechanics of Orbit Using Python(MOPy)}}}\\ \vspace{0.5em}
\normalsize{Submitted to}\\ \vspace{1em}
\includegraphics[scale=0.7]{AU.png}\\ \vspace{1em}
\normalsize{In partial fulfilment of the requirements for the award of degree} \\
\Large{\textbf{Bachelor of Technology}}\\
\Large{\textbf{In}}\\
\Large{\textbf{Aerospace Engineering}}\\ \vspace*{1em}
\normalsize Submitted by: \vspace*{1em} \\
\normalsize
\begin{tabular}{cc}
Ramkiran L. & Manjunath \\ 
17030141AE007 & 17030141AE009 \\ 
\href{mailto:lramkiranBTECH17@ced.alliance.edu.in}{{\color{blue}{\fontfamily{ptm}\selectfont
lramkiranBTECH17@ced.alliance.edu.in}}} & \href{mailto:manjunathBTECH17@ced.alliance.edu.in}{{\color{blue}{\fontfamily{ptm}\selectfont
manjunathBTECH17@ced.alliance.edu.in}}} \vspace*{1em} \\ 
Monisha Patel A. & Thoshitha R. Kumar \\ 
17030141AE013 & 17030141AE027 \\ 
\href{mailto:pamonishaBTECH17@ced.alliance.edu.in}{{\color{blue}{\fontfamily{ptm}\selectfont
pamonishaBTECH17@ced.alliance.edu.in}}} & \href{mailto:kuthoshithaBTECH17@ced.alliance.edu.in}{{\color{blue}{\fontfamily{ptm}\selectfont
kuthoshithaBTECH17@ced.alliance.edu.in}}} \vspace*{1em} \\ 
\end{tabular} 
\normalsize
Under the guidance of\\
\textbf{Dr. Gisa G.S.} \\
Assistant Professor\\
Department of Aerospace Engineering,\\
Alliance College of Engineering and Design,\\
Alliance University\\
Bengaluru. \vspace*{1em} \\

\textbf{Department of Aerospace Engineering}\\
\textbf{Alliance College of Engineering and Design}\\
\textbf{Alliance University, Bengaluru - 562106} \\
\textbf{Batch - 2017-21}\\
\textbf{Year - 2021}
\end{center}
\thispagestyle{empty}
\newpage
\begin{center}
\includegraphics[scale=0.7]{AUText.png} \vspace*{2em}\\
\textbf{CERTIFICATE}
\end{center}
This is to certify that \textbf{Mr. Ramkiran L. (17030141AE007), Mr. Manjunath (17030141AE009), Ms. Monisha Patel A. (17030141AE012)} and \textbf{Ms. Thoshitha R. Kumar (17030141AE027)} students of \textbf{Aerospace Engineering, Bachelor of Technology 2017-21} batch at \textbf{Alliance College of Engineering and Design (ACED), Alliance University, Bengaluru} has completed the project report titled \textit{\textbf{\enquote{Mechanics of Orbit using Python}}} under my guidance in partial fulfillment for the award of Bachelor of Technology degree in Aerospace Engineering, Alliance University, Bangalore during the year 2020-2021. \vspace{1em}\\
\begin{center}
\begin{tabular}{K{7cm} K{0.5cm} K{7cm}}
\underline{\hspace{2.5cm}} & & \underline{\hspace{4cm}} \\ 
\textbf{Dr. Gisa G.S} & & \textbf{Dr. Velmurugarajan K.} \vspace{1em}\\ 
Internal Guide & & Head of the Department \\ 
Department of Aerospace Engineering & & Department of Aerospace Engineering \\
ACED, Alliance University & & ACED, Alliance University \\ 
Bengaluru & & Bengaluru \vspace{3em}\\ 
\multicolumn{3}{c}{\underline{\hspace{3cm}}}\\ 
\multicolumn{3}{c}{\textbf{Dr. Reeba Korah}}\vspace{1em} \\ 
\multicolumn{3}{c}{Interim Dean} \\ 
\multicolumn{3}{c}{Department of Aerospace Engineering}\\
\multicolumn{3}{c}{ACED, Alliance University}\\ 
\multicolumn{3}{c}{Bengaluru}\vspace{2em}\\ 
\end{tabular} 
\end{center}
\textbf{External Viva}
\begin{center}
\begin{tabular}{c K{7cm} K{7cm}}
 & \textbf{Name of Examiners} & \textbf{Signature with date} \\
\textbf{1.} & & \\ 
&&\\
\textbf{2.} & & \\
\end{tabular}
\end{center}
\thispagestyle{empty}
\newpage
\begin{center}
\Large \textbf{DECLARATION}
\end{center}
\normalsize
\hspace{4em}We, Ramkiran L, Manjunath, Monisha Patel A, Thoshitha R. Kumar students of 8\textsuperscript{th} Semester Bachelor of Technology in Aerospace Engineering, Alliance College of Engineering and Design (ACED), Alliance University, Bengaluru, hereby declare that the entire project work entitled \enquote{\textbf{Mechanics of Orbit using Python}} is an authentic record of the work that has been carried out independently by us during final year of our B.Tech at ACED, under the esteemed guidance \textbf{Dr. Gisa G.S}, Assistant Professor, Department of Aerospace Engineering, Alliance college of Engineering and Design, Alliance University. \par

This project report is submitted in partial fulfillment of requirements for the award of the degree of Bachelor of Technology in Aerospace Engineering. The results embodied in this dissertation are original and it has not been submitted in part or full for any degree in any University. \vspace{4em}\\
\textbf{Place:} Bengaluru\\
\textbf{Date:}17/062021 \vspace{5em}\\

\begin{center}
\begin{tabular}{K{7.5cm} K{7.5cm}}
\underline{\hspace{2.5cm}} & \underline{\hspace{2.5cm}} \\ 
\textbf{Ramkiran.L} & \textbf{Manjunath} \\ 
\textbf{17030141AE007} & \textbf{17030141AE009} \vspace{3em}\\ 
\underline{\hspace{3cm}} & \underline{\hspace{4cm}} \\ 
\textbf{Monisha Patel A.} & \textbf{Thoshitha R. Kumar} \\ 
\textbf{17030141AE012} & \textbf{17030141AE027}
\end{tabular} 
\end{center}
\thispagestyle{empty}
\newpage
\begin{center}
\Large \textbf{ACKNOWLEDGEMENT}
\end{center}
\normalsize
\hspace{4em}The satisfaction that accompanies the successful completion of any task would be incomplete without the mention of the people, who are responsible for the completion of the project and who made it possible.\par

We take this opportunity to thank our beloved Interim Dean \textbf{Dr. Reeba Korah}, ACED, Alliance University, Bangalore for providing excellent academic environment in the college and her never--ending support to the B-Tech program.\par

We would like to convey our sincere gratitude to \textbf{Dr. K. Velmurugarajan}, Head of Department of Aerospace Engineering, ACED, Alliance University, Bangalore. \par

We would like to thank our internal guide \textbf{Dr. Gisa G.S}, Assistant Professor, Department of Aerospace Engineering, ACED, Alliance University, Bangalore for her support and encouragement given to carry out the project. \par

We would also like to thank our college staff members and well-wishers who directly or indirectly helped, motivated to complete this project successfully. \par

Lastly, we thank God almighty, our family, professors and friends for their constant encouragement without which this project would not have been possible.\\
\thispagestyle{empty}
\newpage

\begin{abstract}
Orbital mechanics is the study of the motions of artificial satellites and space vehicles moving under the influence of forces. It plays a vital role in planning space missions in various aspects such as designing orbital trajectory for various missions, calculating lagrangian points etc., and can find many engineering applications  including ascent trajectories, reentry and landing, rendezvous computations, lunar and interplanetary trajectories. 

In recent years, as there is a lot of progress unfolding in space industry, many aspiring students are keen on gaining knowledge and pursue careers in space industry. As such, strong foundations of the fundamentals are required for the students for them to get ahead in the field. There are lots of tools like STK, FreeFlyer etc., for learning but these don't start from the bare minimum of the concepts. This led to the idea of developing of MOPy.

Mechanics of Orbit using Python(MOPy) is learning tool designed and developed with the purpose of introducing the core concepts of Orbital Mechanics. MOPy uses a interactive UI with tool tips and a 3D Environment with an interactive virtual universe. The 3D Environment helps the learner to visualize the concepts in a much accessible and easier way.
\end{abstract}

\thispagestyle{empty}
\newpage
\clearpage
\setcounter{page}{1}
\section{Introduction}
\subsection{About Software}
\subsection{List of Features}
\begin{center}
\begin{table}[H]
{\rowcolors{2}{white}{gray!50}
\begin{tabular}{c|K{7.5cm}}
\hline 
Sl. No & \textbf{Section}\\ 
\hline 
1 & Calculation of Orbital Elements\\ 
\hline 
2 & 2D and 3D orbit \\ 
\hline 
3 & Various Parameters at any given point \\ 
\hline 
4 & 2D and 3D Orbits\\ 
\hline 
5 & Calculation of Julian Day \\ 
\hline 
5 & Euler Angles\\ 
\hline 
6 & Sphere Of Influence \\ 
\hline 
7 & Sensitivity Analysis \\ 
\hline 
8 & Position of one Spacecraft w.r.t Another\\ 
\hline 
9 & Calculation of State and Velocity Vector\\ 
\hline
10 & Orbital Transfer  \\ 
\hline
\end{tabular}}
\caption{\label{tab: features}List of Features present in MOPy}
\end{table}
\end{center}
\subsection{Python Libraries Used}
\begin{enumerate}
\item \textbf{NumPy}:  This brings MATLAB like functionality of using Matrices and their operations to python. This enables us to do a lot of stuff without much hassle.
\item \textbf{SciPy} - This enables us to add many features involving more complex computing scenarios as it has features for scientific and technical computing. For example, it has different kinds of solvers for integration which we can use for solving acceleration vector equation to obtain the position vector for an orbit.
\item \textbf{Matplotlib} - This is a plotting tool that is a extension on NumPy that gives the functionality of plotting many different kinds of graph. This library is somewhat similar to the plotting feature of MATLAB.
\item Panda3D - This is a Game Engine based on C++ that takes in syntax from both C++ and Python. This provides real-time 3D visualizations and simulations based on the code.
\item \textbf{SQLite3} - The entire details of the planetary bodies like the orbital elements, planetary ephemeris and others are stored in a local database. SQLite is used as it enables the offline functionality.
\item \textbf{Qt Deisgner} - This enables MATLAB's App Designer like feature of dragging and dropping the UI elements and creating the GUI. This is based on Qt, a cross platform GUI toolkit developed by the Qt Company
\item \textbf{PyQt5 $\&$ PySide2} - These both are the python binding libraries of Qt.
\item \textbf{PyInstaller} - This library lets us convert our python code(.py) into executable file(.exe)


\end{enumerate}
\subsection{Market Research}
There are mainly two kinds of softwares.
\begin{enumerate}
\item Simulation Based programs like STK, FreeFlyer etc.,
\item Sandbox Based programs like Universe SandBox, Kerbal Space Program etc.,
\end{enumerate}
\hspace{4em}The simulation based programs are mainly used to simulate missions and solve problems based on the instance. Both the applications given in the example are used in the industry for all kinds of missions, ranging from very small scale missions that are performed by the students to complicated missions that are performed by NASA and ISRO.

The Sandbox based programs are the stuff that are run by the physics engine that are baked into it. They use a 3D visualization toolkit or engine which lets the user easily interact with the UI, and change the parameters directly from the environment.
\subsection{Objective}
Investing in Space Exploration
\begin{framed}
\begin{quote}
The revenue generated by the global space industry may increase to more than $\$$1 trillion by 2040.
\end{quote}
\end{framed}
\section{Detailed Explanation of Each Feature}
\subsection{Feature Name}
\subsubsection{Theory}
\subsubsection{Algorithm}
\subsubsection{Front End Development}
\section{Conclusion}
\section{Future Scope}

\begin{thebibliography}{99}
	\bibitem{TB-1}{Bate, Roger R., Donald D. Muller and Jerry E. White},{ \textit{\enquote{Fundamentals Of Astrodynamics}}},{New York, NY},{ Dover Publications},{1971}.
	

\end{thebibliography}
\end{document}
